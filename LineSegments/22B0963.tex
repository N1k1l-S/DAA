\documentclass[english, 11pt]{article}
\usepackage[T1]{fontenc}
\usepackage[latin9]{inputenc}
\usepackage[top = 2cm, left = 2.5cm, right = 2.5cm, bottom = 2cm]{geometry}
\usepackage{enumitem}
\usepackage{amsmath}
\usepackage{amsfonts}
\usepackage{amstext}
\usepackage{mdframed}
\usepackage{graphicx}
\usepackage{bbm}
\makeatletter
\@ifundefined{date}{}{\date{}}
\usepackage{tikz}
\usetikzlibrary{quotes, angles, decorations.markings, intersections}
\usetikzlibrary{calc,patterns,angles,quotes, 3d, intersections, positioning, shapes, automata, positioning}
\usepackage{wasysym}
\makeatother
\usepackage{babel}
\usepackage{color}
\usepackage{hyperref}
\hypersetup{
    colorlinks,
    citecolor=black,
    filecolor=black,
    linkcolor=black,
    urlcolor=black
}

\newcommand{\tbox}[1]{\noindent\fbox{\parbox{\textwidth}{#1}}}

\setlength{\parindent}{0pt}

\begin{document}
\noindent\tbox{
\hfill 10/03/2024
\begin{center}
  \huge Programming Assignment 1 \\ % change lecture number
  \Large CS 218 : Design and Analysis of Algorithms % Short topic description
\end{center}
\textit{Done by}: NIKIL S% <NAME> (<Roll Number>)
}

\section*{Algorithm}

My algorithm is based on divide and conquer method similar to mergesort. So, I decided to name
the two main functions that implement the algorithm as \textit{merge()} and \textit{mergesort()}. 
\textit{mergesort()} function takes in parameters posters vector, starting index and ending index.
It is a recursive function that merges two different outlines every time it calls \textit{merge()}. \textit{merge()} 
takes in parameters two vectors of Points, representing two outlines, and returns a single merged vector of Points. 
The \textit{merge()} function is of \textbf{O(n)} while the \textit{mergesort()} function is called \textbf{O(log(n))} times. 
So, effectively the overall order of the algorithm is \textbf{O(nlog(n))}. \\

The \textit{merge()} function performs different operations to merge the two outlines by operating with two points at a time, 
each from the different outlines. I maintain a variable called \textit{dom} that indicates whether the previously pushed point is of outline1 or outline2.
 The major two cases handled by it are:
\begin{itemize}
    \item If the point to be pushed is of different outline from previous, then it pushes the intersection point and the point.
    \item Else, it pushes only the point.
\end{itemize}

The point to be pushed is determined by the y-value of both the outlines at that particular x-value, the higher value, is pushed.

\section*{Area and Length}
The area and length required are calculated by the functions \textit{area\_of\_outline()} and \textit{lengthCovered\_of\_outline()} respectively.

\section*{Other functions}
To help with the implementation of the algorithm, I have defined other functions like \textit{solve\_pt\_on\_line()}, \textit{intersection\_pt()}, and \textit{remove\_pts()}.
\end{document}
